% -*- compile-command: "latexmk -pvc --shell-escape notes.tex" -*-
\documentclass[10pt]{article}

\usepackage[backend=bibtex,style=numeric]{biblatex}
\usepackage[T1]{fontenc}
\usepackage{mathptmx}
\usepackage{amsmath}
\usepackage{stmaryrd}
\usepackage{amsfonts}
\usepackage{amssymb}
\usepackage{cprotect}
\usepackage{mathtools}
\usepackage{amsthm}
\usepackage{float}
\usepackage{svg}
\usepackage{wrapfig}
\usepackage{fancyhdr}
\usepackage[english]{babel}
\usepackage[bookmarks, hidelinks]{hyperref}
\usepackage[a4paper, total={7in, 9.5in}]{geometry}
\usepackage[mode=buildnew]{standalone}% requires -shell-escape
\usepackage{tikz}
\usepackage{minted}
\usemintedstyle{tango}

\addbibresource{bibliography.bib}

\begin{document}

\pagestyle{fancy}
\fancyfoot{}
\fancyfoot[C]{\thepage}

\fancyhead{}
\fancyhead[LE]{\nouppercase{\rightmark\hfill\leftmark}}
\fancyhead[RO]{\nouppercase{\leftmark\hfill\rightmark}}

\section{Nodal Discontinuous Galerkin Spectral Element Method using Gauss-Lobatto quadrature points}

\subsection{Discontinuous Galerkin formulation}
We wish to solve the following system of conservation laws:
\begin{equation*}
  \begin{cases}
    \partial_t \mathbf{u} + \nabla\cdot \mathbf{f}(\mathbf{u}) = 0 \text{ on }\mathbb{R}^+\times\Omega, \\
    \mathbf{u}(0,\cdot) = \mathbf{u}_0,\text{ on } \Omega,
    \end{cases}
\end{equation*}
where \(\mathbf{f}:\mathbb{R}^m\longrightarrow\mathbb{R}^d\) is the flux and \(\mathbf{u} : \mathbb{R}^+\times\Omega\longrightarrow \mathbb{R}^m\) is the unknowns. We introduce the the mesh defined by:
\begin{equation*}
  \Omega = \bigsqcup_{i=1}^N K_i,
\end{equation*}

for which on each mesh element \(K_i\), we require that the numerical solution solves the variationnal formulation on the finite dimensionnal subspace \(\mathbb{R}_n[X,Y]\) of polynomials of partial degree less than \(n\) given by:
\begin{equation}\label{eq:DG}
  \forall\psi\in\mathbb{R}_n[X_1,\ldots, X_d], \int_{K_i}\partial_t u\psi + \int_{\partial K_i} \widehat{f}(u^\text{int}, u^\text{ext})\cdot n\psi - \int_{K_i} f(u)\cdot\nabla\psi = 0,
  \end{equation}
where \(\widehat{f}: \mathbb{R}^m\times\mathbb{R}^m\longrightarrow \mathbb{R}^d\) is the numerical flux. For the sake of simplicity, we will only treat the case of an scalar conservation law on a quadrilateral uniform 2D grid \(\left(x_{i-\frac12}, y_{j-\frac12}\right)_{1\leq i,j\leq N}\), where \(x_{i-\frac12} = hi, y_{j-\frac12} = jh, N\in\mathbb{N}, h = \frac1N\) and \(K_{i,j} = \left[x_{i-\frac12, i+\frac12}\right]\times \left[x_{j-\frac12}, y_{j+\frac12}\right]\). We decompose the solution on a given basis of \(\mathbb{R}_n[X,Y]\) in the reference element \([-1,1]^2\), \emph{i.e.}

\begin{equation}\label{eq:decomp}
  \forall x\in K_{i,j}, u(x) = u^{i,j}(x) = \sum_{k=1}^{(n+1)^2} u_k^{i,j}\psi_k(\xi(x,y), \eta(x,y)),
\end{equation}

where \((x,y)\ni K_{i,j} \longmapsto (\xi, \eta) [-1,1]^2\) is a \(\mathcal{C}^1\)-diffeomorphism between \(K_{i,j}\) and the reference element \([-1,1]^2\). In the present case of an uniform grid of side length \(h>0\), we have:
\begin{equation}\label{eq:transformation}
  \begin{cases}
    \xi = 2(x-x_{i-\frac12})/h-1, \\
    \eta = 2(y-y_{j-\frac12})/h-1.
  \end{cases}
\end{equation}
Expressing the unknown in the reference element \([-1,1]^2\) is done in order to compute integral with a quadrature rule. By plugging (\ref{eq:decomp}) into (\ref{eq:DG}), we get:

\[\forall \ell\in \llbracket 1, (n+1)^2\rrbracket, \sum_{k=1}^{(n+1)^2}\left(\int_{K_i} \psi_k\psi_\ell\right) \frac{du_k^{i,j}}{dt} - \int_{K_i} f(u)\cdot\nabla\psi_\ell + \int_{\partial K_i} \widehat{f}(u^-, u^+)\cdot n\psi_\ell = 0\]
which can be written in matrix form as:
\[M_{i,j}\frac{du^{i,j}}{dt} - S_{i,j}^xf(u) - S_{i,j}^yg(u) + \sum_{e\in\partial K_{i,j}}\int_e\widehat{f}(u^-, u^+)\cdot n\psi_\ell = 0,\]
with mass matrix and stiffness matrix in the reference domain and the physical domain:

\begin{minipage}{.49\linewidth}
  \centering
  \begin{align*}
  [M]_{k,\ell} &= \int_{[-1,1]^2}\psi_k\psi_\ell d\xi d\eta, \\
  [S^\xi]_{k,\ell} &= \int_{[-1,1]^2}\psi_\ell\partial_\xi\psi_kd\xi d\eta, \\
  [S^\eta]_{k,\ell} &= \int_{[-1,1]^2}\psi_\ell\partial_\eta\psi_kd\xi d\eta, \\
  \end{align*}
\end{minipage}%
\begin{minipage}{.49\linewidth}
  \centering
  \begin{align*}
  [M_{i,j}]_{k,\ell} &= \int_{K_{i,j}}(\psi_k\psi_\ell)(\xi,\eta)dxdy, \\
  [S_{i,j}^x]_{k,\ell} &= \int_{K_{i,j}}(\psi_\ell\partial_x\psi_k)(\xi,\eta) dxdy, \\
  [S_{i,j}^y]_{k,\ell} &= \int_{K_{i,j}}(\psi_\ell\partial_y\psi_k)(\xi,\eta) dxdy, \\
  \end{align*}
\end{minipage}

and \(f(u), g(u)\) the component of the functions on \(K_{i,j}\) in the basis \((\psi_k)_{1\leq k\leq (n+1)^2}\).

\subsection{Gauss-Lobatto integral quadrature}
We use 2D the tensor basis of \(\mathbb{R}_n[X,Y]\) generated by the 1D Gauss-Lobatto Lagrange interpolation polynomials \((\psi_k=\ell_p\otimes \ell_q)_{1\leq p,q\leq n+1}\) where:
\[\ell_p = \frac{\prod_{q\neq p} (X-\xi_p)}{\prod_{q\neq p} (\xi_p-\xi_q)},\text{ and }k=p+(n+1)\times (q-1),\]
and \(\xi_1 = -1 < \cdots < \xi_{n+1} = 1\) with \((\xi_2,\ldots, \xi_n)\) the roots of the Legendre polynomial \(\widetilde{L}_{n-1}\). The quadrature rule is given by:
\begin{equation}\label{eq:quadrature}
  \int_{[-1,1]^2} g = \sum_{1\leq p,q\leq n+1} \omega_p\omega_q g(\xi_p, \eta_q).
\end{equation}
This formula is exact for polynomials of partial degree less than \(2n-1\). We can also compute the trace using interpolation points in \(\partial([-1,1]^2)\) needed to compute numerical fluxes. This is one of the advantages of using an quadrature rule with points on the boundary. On the contrary, if we were to use the Gauss-Legendre quadrature rule, we would get an exact quadrature rule for polynomials of degree at most \(2n+1\) (which would make the lagrange interpolation polynomial on those points orthogonal and thus the mass matrix computation exact yielding a diagonal matrix) but we would need to extrapolate outside the convex hull of the interpolation nodes to compute integral on the boundary of elements. Using the transformation to the reference cell element \([-1,1]^2\), we can thus numerically integrate functions defined on the physical element \(K_{i,j}\) by:
\[\int_{K_{i,j}} g(x,y) dxdy = \int_{[-1,1]^2} g(x(\xi,\eta), y(\xi, \eta)) \left|\frac{\partial (x,y)}{\partial (\xi,\eta)}\right| d\xi d\eta = J_h\sum_{1\leq p,q\leq n+1} \omega_p\omega_qg(x(\xi_p,\eta_q), y(\xi_p, \eta_q)),\]
where \(J_h = \frac{h^2}4\) is the Jacobian determinant of the transformation defined in (\ref{eq:transformation}). All of the matrices in the discontinuous Galerkin method are computed using the quadrature (\ref{eq:quadrature}). We get:

\begin{align*}
  &M = \operatorname{Diag}(\omega_1,\ldots,\omega_{n+1})\otimes\operatorname{Diag}(\omega_1,\ldots,\omega_{n+1}),\quad M_{i,j} = \frac{h^2}4 M \\
  &D = [\ell_p'(\xi_q)]_{1\leq p,q\leq n+1}, \\
  &D^\xi = D\otimes I_{n+1},\quad S^\xi = D^\xi M, \quad S^x_{i,j} = \frac h2 S^\xi \\
  &D^\eta = I_{n+1}\otimes D,\quad S^\eta = D^\eta M, \quad S^y_{i,j} = \frac h2 S^\eta
\end{align*}
where \(\otimes\) is the Kronecker product, \(I_{n+1}\) the \(\mathbb{R}^{(n+1)\times(n+1)}\) identity matrix and \(D\) is the 1D differentiation matrix. The stiffness matrix expression is obtained by noticing that:

\begin{align*}
  [S^x_{i,j}]_{k,\ell} &= \int_{K_{i,j}}\psi_\ell(\xi,\eta)\left[(\partial_\xi\psi_k)(\xi,\eta)\frac{\partial\xi}{\partial x}(x,y)+(\partial_\eta\psi_k)(\xi,\eta)\frac{\partial\eta}{\partial x}(x,y)\right]dxdy \\
                       &= \left|\frac{\partial (\xi,\eta)}{\partial (x,y)}\right|\int_{K_{i,j}}\psi_\ell(\xi,\eta)\left[(\partial_\xi \psi_k)(\xi,\eta) \frac{\partial y}{\partial\eta}(\xi,\eta) - (\partial_\eta\psi_k)(\xi,\eta)\frac{\partial y}{\partial\xi}(\xi,\eta)\right] \left|\frac{\partial(x,y)}{\partial(\xi,\eta)}\right|d\xi d\eta \\
                       &= \int_{K_{i,j}}\psi_\ell\left[(\partial_\xi \psi_k) \frac{\partial y}{\partial\eta} - (\partial_\eta\psi_k)\frac{\partial y}{\partial\xi}\right]d\xi d\eta = \frac h2 [S^\xi]_{k,\ell}.
\end{align*}

It is of interest to note that as \(\ell_p'\ell_q\) is of degree \(2n-1\), the quadrature rule for the stiffness matrix is exact. The differentiation matrix is given by:

\begin{align*}
  \ell_p'(\xi_q) =
  \begin{dcases}
    \sum_{m\neq p}\frac1{\xi_p-\xi_m} & \text{if } p=q, \\
    \frac{\prod_{m\not\in\{p,q\}} (\xi_q-\xi_m)}{\prod_{m\neq p}(\xi_p-\xi_m)} & \text{otherwise.}
  \end{dcases}
\end{align*}

Using 1D quadrature rule, we can compute the numerical flux by:

\begin{align*}
  \sum_{e\in\partial K_{i,j}}\int_e \widehat{f}(u^-, u^+)\cdot n\underbrace{\ell_p(\xi)\ell_q(\eta)}_{\psi_k(\xi,\eta)} &= \left(\delta_{p,n+1}\widehat{f}\left(u\left(x_{i+\frac12}^-, y_j\right), u\left(x_{i+\frac12}^+, y_j\right)\right)-\delta_{p,1}\widehat{f}\left(u\left(x_{i-\frac12}^-, y_j\right), u\left(x_{i-\frac12}^+, y_j\right)\right)\right) \\
  &+\left(\delta_{q,n+1}\widehat{g}\left(u\left(x_i-, y_{j+\frac12}^-\right), u\left(x_i, y_{j+\frac12}^+\right)\right)-\delta_{q,1}\widehat{g}\left(u\left(x_i, y_{j-\frac12}^-\right), u\left(x_i, y_{j-\frac12}^+\right)\right)\right),
\end{align*}

where \(y_j = y(\xi_j)\) and \(x_i = x(\xi_i)\).

\subsection{Time discretization}
So far, we have not dealt with the time discretization. We use a method of lines approach: we first discretize in space resulting in a system of ODE's to integrate. Indeed, we can write the discontinuous galerkin flux approximation as:

\[\frac{du}{dt} = L_h(u),\]

where \(u\) is a vector comprised of all the degrees of freedom of the space discretization and \(L_h\) is the discrete operator representing \(-\nabla\cdot f(u)\). As we use a high order space discretization, it would be better to use a high order time discretization as well. The most common time discretization is the SSP-RK (strong stability preserving Runge Kutta) method. It guarantees that if an explicit Euler timestep if stable for the semi-norm \(\|\cdot\|\) (\emph{i.e.} \(\|u+\delta tL_h(u)\| \leq \|u\|\) for \(\delta t \leq \Delta t\) and all \(u\)), then the SSP-RK satisfies a similar CFL stability condition for this semi-norm of the form \(\delta t \leq c\Delta t\) with \(c\in [0,1]\). Such high order SSP-RK methods can be found that maximizes \(c=1\), \emph{e.g.}

\begin{align*}
  \begin{dcases}
    u_h^{(1)} = u_h^n + \Delta t L_h(u_h^n) \\
    u_h^{(2)} = \frac34 u_h^n + \frac14 u_h^{(1)} + \frac14\Delta tL_h(u_h^{(1)}) \\
    u_h^{n+1} = \frac13 u_h^n + \frac23 u_h^{(2)} + \frac23 \Delta tL_h(u_h^{(2)}).
  \end{dcases}
\end{align*}

However, for DG methods have been found to be \(L^\infty(0,T;L^2(\mathbb{R}))\) unstable \cite{Cockburn1998} for the Euler forward step so that SSP-RK would not carry out any stability property \emph{a priori}. Nevertheless, it has been shown numerically that using a SSP-RK of order \((n+1)\) is stable for a CFL condition of the form:
\[c\frac{\Delta t}{\Delta x} \leq \frac1{2n+1}.\]

\subsection{Generalized slope limiter}

\subsubsection{Modal and nodal basis}
In this subsection, we will only treat the 1D case. To limit spurious oscillations, we need to introduce a slope limiter that will make the numerical scheme TVDM (total variation diminishing in the mean). However, as we chose a nodal local basis, we need to switch to a modal basis in order to gain information about the slope of the solution. We thus need the compute the change of basis matrix from the Lagrange interpolation polynomials nodal basis \(\{\ell_1,\ldots, \ell_{n+1}\}\) to the Legendre polynomial modal basis \(\{L_1,\ldots, L_{n+1}\}\). In order to do so, we first express the Legendre polynomial basis in the canonical basis \(\{1, X, \ldots, X^n\}\) of \(\mathbb{R}_n[X]\). If we introduce the following positive matrix matrix:

\begin{align*}
  M_{k,\ell} = \int_{-1}^1 X^k X^\ell = \frac{1+(-1)^{k+\ell}}{k+\ell+1},
\end{align*}

representing the scalar product of the canonical basis, we can compute an orthogonal polynomial basis of \(\mathbb{R}_n[X]\) by orthogonalizing the canonical basis \(\{e_1,\ldots,e_{n+1}\}\) with respect to the scalar product \((x,y)\mapsto x^TMy\). This is equivalent to computing the QR decomposition of the matrix \(M^\frac12=QR\) yielding \(R^{-1}\) to be an orthonormal basis of \(\mathbb{R}_n[X]\) expressed in the canonical basis. This is the Legendre polynomials up to a normalization factor. If we introduce the vandermonde matrix:

\begin{align*}
  V = \mathcal{M}_{\{1,X,\ldots, X^n\}}^{\{\ell_1,\ldots, \ell_{n+1}\}} = [\xi_i^{j-1}]_{1\leq k,\ell\leq n+1}.
\end{align*}
By change of basis matrix, we mean that if \(P = \sum_{i=1}^{n+1}\beta_i X^{i-1}\in\mathbb{R}_n[X]\) and \(\mathbb{\beta} = [\beta_1 \cdots \beta_{n+1}]^T\), the if we set \(\alpha = [\alpha_1 \cdots \alpha_{n+1}]^T = \mathcal{M}_{\{1,X,\ldots,X^n\}}^{\{\ell_1,\ldots, \ell_{n+1}\}}\beta\) we have \(P = \sum_{i=1}^{n+1}\alpha_i\ell_i\). We can compute the normalizing factors \([1\quad \xi_{n+1} \cdots \xi_{n+1}^n]R^{-1} = [1 \cdots 1]R^{-1}\) and thus get the Legendre polynomials:
\begin{align*}
  \mathcal{M}^{\{1, X, \ldots, X^n\}}_{\{L_1,\ldots, L_{n+1}\}} = R^{-1}\operatorname{Diag}([1\cdots 1]R^{-1})^{-1}
\end{align*}

Using the previously two defined matrices, we can get the nodal to modal change of basis matrix:

\begin{align*}
  \mathcal{M}_{\{\ell_1,\ldots,\ell_{n+1}\}}^{\{L_1,\ldots, L_{n+1}\}} &= \mathcal{M}_{\{1,X,\ldots, X^n\}}^{\{L_1,\ldots, L_{n+1}\}}\mathcal{M}_{\{\ell_1,\ldots,\ell_{n+1}\}}^{\{1, X,\ldots, X^n\}} \\
                                                                       &= \left[\mathcal{M}_{\{1,X,\ldots, X^n\}}^{\{\ell_1,\ldots, \ell_{n+1}\}}\mathcal{M}_{\{L_1,\ldots,L_{n+1}\}}^{\{1, X,\ldots, X^n\}}\right]^{-1}
\end{align*}

\subsubsection{TVDM scheme}

\printbibliography

\end{document}
